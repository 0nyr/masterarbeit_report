\chapter{Background}\label{sec:background}

% Introduce the related state-of-the-art and background information in order to understand the method developed in the thesis. 

The evolving landscape of cybersecurity necessitates robust techniques for safeguarding digital communications. OpenSSH, a pivotal element in this landscape, is a popular implementation of the Secure Shell (\acrshort{ssh}) protocol, which enables secure communication between two networked devices. The protocol is widely used in the industry, particularly in the context of remote access to servers. Using digital forensic techniques, it is possible to extract the SSH keys from memory dumps, which can then be used to decode encrypted communications thus allowing the monitoring of controlled systems. At the crux of this Masterarbeit is the development of machine learning algorithms to predict SSH keys within these heap dumps, focusing on using graph-like-structures and vectorization for custom embeddings. With an interdisciplinary approach that fuses traditional feature engineering with graph-based methods as well as memory modelization for inductive reasoning and learning inspired by recent developments in \acrfull{kg}s, this research not only leverages existing machine learning paradigms but also explores new avenues, such as \acrfull{gcn}.

The objective of this background section is multifaceted. Since the project has seen two distinct phases, one more classical \acrfull{ml} approach, and a second one centered aroung graph-based advanced learning methods, the background section is divided into several subsections that introduce the reader to the different concepts and techniques used in the project. First, it aims to offer an overview of SSH protocols, particularly focusing on OpenSSH key implementations \autoref{sec:background:ssh}. Second, it delineates the dataset and prior work on key extraction techniques including SmartKey \autoref{sec:background:kex}. Third, it delves into the technical aspects of graphs modelization \autoref{sec:background:graph}, followed by feature engineering and embeddings both traditional and graph-based \autoref{sec:background:processing}. Finally, it addresses the machine learning models employed in the research, emphasizing their suitability for maximizing recall in key prediction \autoref{sec:background:ml}. By fusing these distinct but interrelated areas, this section lays the foundation for the research methodologies and hypotheses tested in this study.

\section{SSH and OpenSSH Implementation}\label{sec:background:ssh}

    \subsection{Basics of the Secure Shell Protocol (SSH)}
    
    The Secure Shell Protocol, commonly known as \acrshort{ssh}, is designed to facilitate secure remote login and other secure network services over insecure networks. \acrshort{ssh} has been design since its inception with security in mind, as a successor of the Telnet protocol, which is not secure, and other \say{unsecured remote shell protocols such as rlogin, rsh and rexec} \cite{SSHkex22}. 
    
    \subsubsection{SSH design and origin}
    As stated by the authors of the \citetitle{SSHReport18}, \say{The founder of SSH, Tatu Ylönen, designed the first version of the SSH protocol after a password-sniffing attack at his university network. Tatu released his implementation as freeware in July 1995, and the tool quickly gained in popularity. Towards the end of 1995, the SSH user base had grown to 20,000 users in fifty countries. By 2000, there were an estimated 2,000,000 users of the protocol. Today, more than 95\% of the servers used to power the Internet have SSH installed in them. The SSH protocol is truly one of the cornerstones of a safe Internet.} \cite{SSHReport18}.

    \acrshort{ssh} is defined in \citetitle{RFC4251} \cite{RFC4251}. It is divided into three major components:

    \begin{itemize}
        \item \textbf{Transport Layer Protocol:} This provides server authentication, confidentiality, and integrity. It can also optionally provide compression. Typically, the transport layer runs over a TCP/IP connection but can also be used on top of any other reliable data stream.
        \item \textbf{User Authentication Protocol:} Running over the transport layer, this protocol authenticates the client-side user to the server. Multiple methods of authentication such as password and public key are supported.
        \item \textbf{Connection Protocol:} This multiplexes the encrypted tunnel established by the preceding layers into several logical channels. Channels can be used for various purposes, such as setting up secure interactive shell sessions or tunneling arbitrary TCP/IP ports.
    \end{itemize}

    \say{The client sends a service request once a secure transport layer connection has been established. A second service request is sent
    after user authentication is complete. This allows new protocols to be defined and coexist with the protocols listed above} \cite{RFC4251}.

    \subsubsection{SSH keys}
    For the scope of this Masterarbeit, understanding SSH's key exchange and encryption mechanism is important. As Summarized in SSHKex \cite{SSHkex22}, the SSH key exchange procedure results in a derived master key \textit{K} and a hash value \textit{h}. These are critical for client-server communication encryption and session identification.

    During the key exchange, multiple session keys are computed for different purposes:

    \begin{itemize}
        \item \textbf{Initialization Vectors:} Key A and Key B are used for initialization vectors from the client to the server and vice versa.
        \item \textbf{Encryption Keys:} Key C and Key D serve as encryption keys for client-to-server and server-to-client communications, respectively.
        \item \textbf{Integrity Keys:} Key E and Key F are used to maintain the integrity of the data transmitted between the client and server.
    \end{itemize}

    \subsubsection{SSH key encryption}
    These keys are computed using hash functions that take the master key \textit{K} and a hash value \textit{h}, a unique letter (A, B, C, D, E, or F), and the session ID as inputs. This is summarized in \citetitle{OpenSSHUnderHood07}:
    \say{
        The equations used for deriving the above vectors and keys are taken from RFC 4253 \cite{RFC4253}. In the following, the $||$ symbol stands for concatenation, K is encoded as mpint, $"A"$ as byte and $session_id$ as raw data. Any letter, such as the $"A"$ (in quotation marks) means the single character A, or ASCII 65.
        \begin{itemize}
            \item Initial IV client to server: $HASH(K || H || "A" || session_id)$.
            \item Initial IV server to client: $HASH(K || H || "B" || session_id)$.
            \item Encryption key client to server: $HASH(K || H || "C" || session_id)$.
            \item Encryption key server to client: $HASH(K || H || "D" || session_id)$.
            \item Integrity key client to server: $HASH(K || H || "E" || session_id)$.
            \item Integrity key server to client: $HASH(K || H || "F" || session_id)$.
        \end{itemize}
    } \cite{OpenSSHUnderHood07}. Details about the hash function are given in the next section.
    
    The most interesting keys are the encryption keys, as they are used to encrypt the communication between the client and the server. The other keys are used for integrity checks and initialization vectors. Decrypting encrypted SSH communication necessitates either to retreive these session keys and variables so as to recompute the keys, or to retreive those keys directly, which is the focus of this Masterarbeit. 

    \subsection{OpenSSH Implementation}
    OpenSSH (OpenBSD Secure Shell) is an open-source implementation written in C of the SSH protocol suite, and it is the most widely used SSH implementation \cite{OpenSSHUnderHood07}. It is the default SSH implementation on most Linux distributions, and it is also available for Windows. OpenSSH is used for a wide range of purposes, including remote command-line login and remote command execution. It is also used for port forwarding, tunneling, and transferring files via SCP and SFTP either manually or via automated processes, such as backup systems, configuration management tools, and automated software deployment tools. 

    \subsubsection{OpenSSH components}
    OpenSSH is composed of several tools and daemons, including client and server components \cite{PortableOpenSSHGitHub}:
    \begin{itemize}
        \item \textbf{ssh:} The basic client program that allows to log into and execute commands on a remote machine.
        \item \textbf{scp:} A program for securely copying files between machines.
        \item \textbf{sftp:} An interactive file transfer program that uses SSH to secure the connection.
        \item \textbf{sshd:} This is the SSH daemon that runs on the server. This is used for connecting to a remote machine when using the SSH client from another system.
        \item \textbf{ssh-agent:} The program that holds private keys in memory, so one doesn't have to enter ones passphrase every time.
        \item \textbf{ssh-add:} A program for adding RSA or DSA identities to the authentication agent.
        \item \textbf{ssh-keygen:} A utility for creating and managing SSH keys.
        \item \textbf{ssh-keyscan:} A utility for gathering public SSH host keys from a number of hosts.
        \item \textbf{ssh-keychk:} A utility for checking the validity of SSH keys.
        \item Several other tools to support the SSH protocol and the OpenSSH implementation.
    \end{itemize}

    \subsubsection{OpenSSH hashing}
    OpenSSH employs a variety of hash functions and algorithms to secure data, most commonly using SHA1. However, SHA1 is increasingly seen as weak due to its vulnerability to collision attacks \cite{OpenSSHUnderHood07}. In light of this, the contemporary standard leans towards SHA512. The hash functions are used alongside cipher algorithms like \say{Advance Encryption Standard (AES) Cipher Block Chaining (CBC), AES Counter (AES-CTR), and ChaCha20} \cite{SmartKex22}. The Message Authentication Code (MAC) typically uses either MD5 or SHA1 hash algorithms in combination with a secret key. Since cybersecurity and cryptography are constantly evolving, so do \acrshort{ssh} and OpenSSH. Depending on the version \cite{OpenSSHUnderHood07}, the available hash options include:
    
    \begin{itemize}
        \item \textbf{ssh-dss:} \textit{(disabled at run-time since OpenSSH 7.0 released in 2015)} SSH-1 version using Digital Signature Algorithm (DSA) from the Digital Signature Standard (DSS). Originally popular but phased out due to vulnerabilities to collision attacks for DSA Key in a 1024-bit modulus. As stated by \citetitle{RFC9142}: \say{These attacks are still computationally very difficult to perform, but it is desirable that any key exchange using SHA-1 be phased out as soon as possible} \cite{RFC9142} \cite{OpenSSHReleaseNotes7-0}.
        
        \item \textbf{ssh-rsa:} \textit{(disabled at run-time since OpenSSH 8.8 released in 2021)} It refers to the use of RSA (Rivest-Shamir-Adleman) encryption algorithm. In the context of SSH-1, this version had to be replaced due to the related to key size issue similar to DSS: \say{RSA 1024-bit keys have approximately 80 bits of security strength}... \say{which may not be sufficient for most users.} \cite{RFC9142} \cite{OpenSSHReleaseNotes8-8}.
        
        \item \textbf{ecdsa-sha2-nistp256:} \textit{(since OpenSSH 5.7 released in 2011)} Uses the SHA-2 family for hashing and the NIST P-256 curve. It is considered secure and efficient, with an \acrfull{ess} of 128 bits \cite{RFC9142} \cite{OpenSSHReleaseNotes5-7}.
        
        \item \textbf{ecdsa-sha2-nistp384:} \textit{(since OpenSSH 5.7)} Utilizes the SHA-2 family and the larger NIST P-384 curve for additional security at the cost of performance. It has an \acrshort{ess} of 192 bits \cite{RFC9142} \cite{OpenSSHReleaseNotes5-7}.
        
        \item \textbf{ecdsa-sha2-nistp521:} \textit{(since OpenSSH 5.7)} Employs SHA-2 and the even larger NIST P-521 curve for maximal security with an \acrshort{ess} of 256 bits \cite{RFC9142}. It is less commonly used due to performance considerations \cite{OpenSSHReleaseNotes5-7}. 
        
        \item \textbf{ssh-ed25519:} \textit{(since OpenSSH 6.5 released in 2014)} Known for high security and performance efficiency; employs the Ed25519 elliptic curve with an \acrshort{ess} of 128 bits \cite{RFC9142} which is similar to $ecdsa-sha2-nistp256$, and has been more prevalent following the 2013 suspicions of NSA backdoors in NIST curves \cite{Adamantiadis2013} following the Snowden revelations \cite{NSAFoilSafeguards2013} \cite{GuardianEncryption2013} \cite{OpenSSHReleaseNotes6-5}.
        
        \item \textbf{rsa-sha2-256:} \textit{(since OpenSSH 7.2 released in 2016)} An upgrade from ssh-rsa, using SHA-256 (with \acrshort{ess} of 128 bits) for hashing to improve security without major performance hits \cite{OpenSSHReleaseNotes7-2}.
         
        \item \textbf{rsa-sha2-512:} \textit{(since OpenSSH 7.2)} Similar to $rsa-sha2-256$ but employs SHA-512 for even stronger security, albeit with some performance cost \cite{OpenSSHReleaseNotes7-2}.
        
        \item \textbf{ecdsa-sk:} \textit{(since OpenSSH 8.2 released in 2020)} Security Key-enabled, uses NIST curves and is geared towards modern hardware-based authentication \cite{OpenSSHReleaseNotes8-2}.
        
        \item \textbf{ed25519-sk:} \textit{(since OpenSSH 8.2)} Similar to ssh-ed25519 but integrates hardware-based Security Keys for an additional layer of security \cite{OpenSSHReleaseNotes8-2}.
        
        \item \textbf{NTRU Prime-x25519:} \textit{(since OpenSSH 9.0)} A new, highly secure algorithm focused on post-quantum cryptography, providing future-proof security \cite{NTRUPostQuantum17} \cite{OpenSSHReleaseNotes9-0}.
    \end{itemize}
    
    These hashes have fixed lengths such that key lengths range between 12 and 64 bytes \cite{SmartKex22}. Since high-quality random number generation is crucial to ensure that thoses keys are secure and difficult to predict, it can thus be assumed that thoses key have a high entropy \cite{McLaren2019}. This is a crucial assumption as it is the basis for the use of both brute force and machine learning algorithms to predict the presence and location of SSH keys in memory dumps.

    The keys generated by these hash functions are pseudo-random numbers stored in the system's RAM. Following the Kerckhoffs' principle: that \say{a cryptosystem should be secure, even if everything about the system, except the key, is public knowledge}, the code for the OpenSSH implementation is open-source and available on GitHub \cite{PortableOpenSSHGitHub}. This allows for the analysis of the code and the identification of the memory structures where the keys are stored.

    \subsection{The state of SSH security}
    Since its origins, SSH has been developed with cybersecurity in mind, and is generally considered a secure method for remote login and other secure network services over an insecure network. However, as with any technology, it can be exploited if not configured or managed correctly. The protocol is used by system administrators to manage remote systems, and it is also used by automated processes to transfer data and perform other tasks. This makes SSH a valuable target for attackers. In fact, SSH has been a popular target for cyber-attacks. Due to being so prevalent, it is often used by threat actors either as a vector for initial access, as a means to move laterally across a network or as a covered exit for exfiltration of sensitive data \cite{APTTactics19}. The encrypted nature of its communications makes it an attractive option for attackers, as it can be difficult to detect malicious activity.
    
    \subsubsection{SSH security issues}
    Here are some cases where SSH can involved in cyber-attacks, although it's important to note that SSH itself is not inherently insecure:
    \begin{itemize}
        \item \textbf{SSH Brute-Force Attacks:} One of the most common types of attacks involving SSH is a brute-force attack, where an attacker tries to gain access by repeatedly attempting to login with different username-password combinations. These attacks are not sophisticated but can be effective if strong authentication measures are not in place. For instance, the botnet \textit{Chabulo} was used to launch a large-scale brute-force attack \say{through compromised SSH servers and IoT devices} in 2018 \cite{SSHReport18}. 
        \item \textbf{SSH Key Theft:} In some advanced attacks, threat actors have stolen SSH keys to move laterally across a network after initial entry. This allows them to authenticate as a legitimate user and can make detection much more challenging. It can \say{ occurs when users have their SSH password or unencrypted keys stolen through a variety of methods (sniffed via a key-logging console program, shoulder-surfed via bad security awareness, poor key management practices, etc.).} \cite{SSHIdentityTheft05}.
        \item \textbf{Man-in-the-Middle Attacks:} Although SSH is designed to be secure, it can be susceptible to man-in-the-middle attacks if proper verification of SSH keys is not done during the initial connection setup \cite{OpenSSHUnderHood07}.
        \item \textbf{Misconfiguration:} As with any technology, misconfiguration can lead to security issues. For example, leaving default passwords, using weak encryption algorithms, or enabling root login can all make an SSH-enabled system vulnerable \cite{SSHBotnetInfect21}.
    \end{itemize}

    \subsubsection{SSH vulnerabilities}
    In cybersecurity, it is generally considered that any system that is connected to the Internet will be attacked at some point. Similarly, it is a common saying that no system is 100\% secure. This is true for SSH as well. Although it is a secure protocol, it can be exploited if not configured or managed correctly. 
    
    Some vulnerabilities have also been discovered in the protocol itself, although these are rare.
    \begin{itemize}
        \item \textbf{SSH-1 Vulnerabilities:} A series of vulnerabilities in the first implementation of SSH were discovered from 1998 to 2001, with its subsequent fixes leading to unauthorized content insertion and arbitrary code execution. SSH-1 had many design flows and is now considered obsolete. \cite{CoreSecurity23}, \cite{SSH1Vulnerability01}. 
        \item \textbf{CBC Plaintext Recovery:} A theoretical vulnerability discovered in 2008 affecting all versions of SSH, allowing the recovery of up to 32 bits of plaintext from CBC-encrypted ciphertext \cite{USCERT2011}.
        \item \textbf{Suspected Decryption by NSA:} Leaked information in 2014 suggested that the NSA might be able to decrypt some SSH traffic, although the protocol itself was not confirmed to be compromised \cite{Spiegel14}.
    \end{itemize}
    
    \subsubsection{SSH and cyber-attacks}
    SSH has been used in many high-profile cyber-attacks and malwares, including the following:
    \begin{itemize}
        \item \textbf{Operation Windigo:} This was a large-scale campaign that infected over 25,000 UNIX servers. SSH was one of the vectors used for maintaining control over compromised servers. A report by ESET mentions that the  OpenSSH backdoor Linux/Ebury was first discovered in 2011 as a component of the aforementioned operation. \say{This operation has been ongoing since at least 2011 and has affected high profile servers and companies, including cPanel - the company behind the famous web hosting control panel - and Linux Foundation's kernel.org - the main repository of source code for the Linux kernel} \cite{ESETWindigo14}. 
        \item \textbf{Linux/Hydra:} Initially unleashed in 2008, this malware is a fast login cracker that targets a range of popular protocols including SSH. Hence, SSH is one of its primary vectors to gain initial access to Internet of Things (IoT) devices. Once a device is infected by Linux/Hydra, it joins an IRC channel and initiates a SYN Flood attack \cite{ClassificationMalware21}.
        \item \textbf{Psyb0t:} Discovered in early 2009, Psyb0t is an IRC-controlled malware specifically designed to target devices with MIPS architecture, such as routers and modems. Notably, it was responsible for orchestrating a DDoS attack against the DroneBL service, infecting up to 100,000 devices for this purpose. The malware is equipped to conduct UDP and ICMP flood attacks and employs a brute-force attack mechanism against Telnet and SSH ports. Remarkably, it uses a pre-configured list of 6,000 usernames and 13,000 passwords to perform these attacks \cite{ClassificationMalware21}.
        \item \textbf{Chuck Noris:} Similar to Psyb0t in its objectives and methods, Chuck Noris targets routers and DSL modems, focusing on SoHo (small office/home office) devices. However, unlike Psyb0t, which uses ICMP flood attacks, Chuck Noris deploys ACK flood attacks. The malware carries out brute-force attacks on Telnet and SSH open ports, drawing parallels to the tactics employed by Psyb0t but with the specific variation in flooding techniques \cite{ClassificationMalware21}.
    \end{itemize}

    It's worth noting that in many of these cases, SSH was not the initial attack vector but was used at some stage in the attack lifecycle. Properly configured and managed SSH is still considered a secure and robust protocol for remote access and data transfer. In all those situations, a tool monitoring the SSH traffic could have detected the malicious activities and prevented the attack.

    \subsection{The Imperative of SSH Honeypots in Cybersecurity Monitoring}
    SH (Secure Shell) has become an indispensable protocol for secure communication but can also conceil malicous agents. This reality underscores the urgency for robust monitoring mechanisms capable of identifying suspicious activities in real-time. Among various countermeasures, SSH honeypots have emerged as a particularly effective tool for monitoring and gathering intelligence on potential threats. 

    An SSH honeypot is a decoy server or service that mimics legitimate SSH services. The primary aim is to attract cybercriminals and study their tactics, thereby offering an active form of surveillance and data collection. Unlike traditional intrusion detection systems, honeypots do not merely identify an attack; they engage the attacker in a controlled environment, enabling detailed observation and logging of the intruder's actions. This allows for the collection of valuable information, such as the attacker's IP address, the tools used, and the techniques employed. This data can then be used to enhance security measures and develop more robust countermeasures \cite{ClassificationMalware21}. 

    SSH honeypots serve as an invaluable asset in the cybersecurity arsenal, providing not just a reactive but a proactive measure against evolving cyber threats. They can collect actionable intelligence on new hacking methods, malware, and exploitation scripts. This information can be crucial for proactively securing actual production environments. The data collected can also be used to trace back to the origin of the attack, facilitating legal pursuits against the perpetrators. By diverting attackers to decoy servers, honeypots also protect real assets from being targeted, saving both computational resources and administrative effort needed for post-incident recovery.

    Popular SSH honeypots include Kippo, Cowrie, and HoneySSH.  Cowrie is a fork of Kippo, with additional features such as logging of attacker's keystrokes and file transfer. 

    \begin{itemize}
        \item \textbf{Kippo:} Kippo is a medium-interaction honeypot that logs the attacker's shell interaction. It specializes in capturing brute force and Telnet-based attacks \cite{ClassificationMalware21}.
        
        \item \textbf{Cowrie:} Serving as Kippo's successor, Cowrie emulates various protocols including SSH, SFTP, and SCP. It logs events in JSON format, making it particularly useful for detecting brute force and Telnet-based attacks, as well as spoofing attacks \cite{ClassificationMalware21}.
        
        \item \textbf{IoTPOT:} This IoT-focused honeypot supports multiple CPU architectures and can detect a variety of attacks including brute force, DoS, and sniffing attacks on Telnet, SSH, and HTTP ports \cite{ClassificationMalware21}.
        
        \item \textbf{HoneySSH:} HoneySSH is a low-interaction honeypot that emulates an SSH server and logs the attacker's IP address, username, and password \cite{honeyssh17}.
        
        \item \textbf{Sarracenia (SSHKex):} Introduced in 2018, Sarracenia is a high-interaction SSH honeypot that has been enhanced by SSHKex. Instead of \say{requiring the VM to be paused for every incoming or outgoing packet, which degrades the server performance} \cite{SSHkex22}, SSHKex allows for the extraction of derived SSH session keys. This reduces the performance degradation significantly, as the VM is paused less frequently \cite{SSHkex22} \cite{SarraceniaSSHHoneypot18}.
    \end{itemize}

    These honeypots are useful tools for gathering intelligence on potential threats. However, they are not without their limitations. For instance, they are not able to mimic a real system, such that attackers might be able to detect them  \citetitle{SSHHoneypotEffectiveness23}. Hence, the need for more advanced SSH honeypots that can leverage data forensic and machine learning techniques so as to be able to use directly a real server as a honeypot, without the need to emulate a system. The current master's thesis is aligned with this ongoing research, further enhancing the state of SSH honeypots.

\section{Prior work on key extraction}\label{sec:background:kex}

    \subsection{SSHKex}\label{sec:background:kex:sshkex}
    
    SSHKex is a research project that aims to address the challenges of analyzing encrypted SSH traffic by leveraging \acrfull{vmi} techniques. Developed by \citeauthor{SSHkex22}, the project focuses on extracting SSH keys and decrypting SSH network traffic in a stealthy, non-intrusive manner while maintaining evidence integrity \cite{SSHkex22}. This paper is itself a continuation of the work presented in \citetitle{SarraceniaSSHHoneypot18} \cite{SarraceniaSSHHoneypot18}, which introduced Sarracenia, a high-interaction SSH honeypot. It is also related to a range of other research projects and papers \cite[section 5.6 and 6]{SSHkex22}.
    
    The SSHKex approach combines standard network traffic capturing methods with dynamic SSH session key extraction. It assumes that the SSH implementation running on the server is known, which is crucial for the key extraction process. The project employs VMI tools like LibVMI and Volatility to gain a complete and untainted view of all guest VM's state information. This allows to efficiently locate SSH session keys in the main memory of a Linux machine. 

    Here is a summary of the SSHKex methodology for key extraction:
    \begin{enumerate}
        \item \textbf{Data Structure Information:} The method leverages detailed knowledge about the data structures used to store the keys. Specific debugging symbols corresponding to the SSH implementation version on the target system provide essential offset values to facilitate the extraction of key material. The structures of interest include \texttt{struct ssh}, \texttt{struct session\_state}, \texttt{struct newkeys}, and \texttt{struct sshenc}. These structures store a range of information such as IP addresses, ports, session states, and encryption keys.

        \item \textbf{Tracing OpenSSH Functions:} Function tracing is employed to identify the precise locations of data structures and to extract keys at the right time. The focus is on two key functions: \texttt{kex\_derive\_keys} (which initiates key generation) and \texttt{do\_authentication2} (which kicks off user authentication).

        \item \textbf{Breakpoints Injection:} Software breakpoints are intentionally placed in the program execution to facilitate debugging. SSHKex utilizes Virtual Machine Introspection (VMI) to inject these breakpoints at the initial points of the two aforementioned key functions.

        \item \textbf{Key Extraction:} Upon calling the \textit{kex\_derive\_keys} function, SSHKex initially stores the address of the \textit{ssh struct}. The actual keys are extracted from memory when the \textit{do\_authentication2} function is subsequently called, adhering to the known structures. 

        \item \textbf{Key Indexing:} OpenSSH stores client-to-server and server-to-client keys in distinct indices of the \texttt{newkeys} structure. SSHKex extracts keys based on these specific indices.

        \item \textbf{Handling Multiple Connections:} To manage multiple SSH connections, OpenSSH spawns child processes. SSHKex extends its key extraction strategy to each child process by identifying them through their unique process IDs.
    \end{enumerate}
    
    One of the key strengths of SSHKex is its focus on stealthiness, preservation, and evidence integrity. The approach aims to be as unobtrusive as possible, avoiding any modifications to the system under investigation. This is particularly important in forensic contexts, where the integrity of the evidence is crucial \cite{SSHkex22}.

    \subsection{SmartKex}

    SmartKex is a direct followup project that focuses on the extraction of SSH keys from heap memory dumps. Its primary objective is to automate the process of SSH key extraction from heap memory dumps. The project introduces a machine learning-assisted methodology that significantly improves the efficiency and accuracy of key extraction compared to traditional brute-force methods. This method is also significantly more straightforward to implement compared to the previous SSHKex approach, which requires detailed knowledge of the SSH implementation and the ability to inject breakpoints into the program execution.

    SmartKex discusses two distinct methods for SSH key extraction:
    \begin{itemize}
        \item \textbf{Brute-Force Baseline Method:} This is a traditional approach that scans through the heap memory to identify potential keys based on known patterns.
        \item \textbf{Machine Learning-Assisted Method:} This  approach uses a Random Forest algorithm trained on a highly imbalanced dataset using \acrshort{smote} balancing. The machine learning model is designed to identify SSH keys with high precision and recall rates, but is not exact as compared to the brute-force method since it is based on a probabilistic model.
    \end{itemize}

        \subsubsection{Baseline brute-force method}

        Here is a summary of SmartKex's brute-force method for SSH key extraction from heap dumps \cite{SmartKex22}:
        \begin{enumerate}
            % complete the following with information from Christopher
            %   * how (which tool) was used to generate the dataset
            %   * what architecture from 
            \item \textbf{Heap Dump Generation:} Heap dump binary files of OpenSSH server process have been generated (ASK HOW) and serves as the input for the key extraction process. The exact process and architecture is not described in SmartKex paper, but we suppose it was done on a \textit{linux-x86\_64} architecture.
            
            \item \textbf{Data Reduction:} To minimize the heap size, the method removes memory pages that are irrelevant (empty) based on Hamming distance.
            
            \item \textbf{Brute-force key search:} Starting from the first byte, a key length of 128 bytes is taken from the heap dump as the potential key. The algorithm iterates over the entire heap, continuously updating the potential key until the heap's end is reached.
            
            \item \textbf{Decryption Attempt:} For every potential key, an attempt is made to decrypt network packets. If decryption fails, the process is repeated with a new potential key.
        \end{enumerate}
        
        Although the brute-force approach is exact, it is computationally expensive. It performs poorly especially when keys are located at the end of the heap dump \cite[section 6.2]{SmartKex22}.

        \subsubsection{SmartKex machine-learning method}

        The real innovation of SmartKex is its machine learning-assisted methodology for SSH key extraction. At the cost fo exactness, this approach is significantly faster than the brute-force method and has a high degree of accuracy in identifying encryption keys. It also allows for the heap size to be reduced to less than 2\% of its original size, further optimizing the extraction process.

        Here is a summary of SmartKex's machine learning-assisted method for SSH key extraction from heap dumps \cite{SmartKex22}:

        \begin{enumerate}
            \item \textbf{Heap Dump inputs:} Similarly to the brute-force method, heap dump binary files of OpenSSH also serve as inputs for the key extraction process.
            \item \textbf{Preprocessing:} The raw heap dump is resized into an $N \times 8$ matrix. High entropy parts of the heap dump, which are likely to be encryption keys, are identified using the logical AND operation on the vertical and horizontal differences of adjacent bytes. This creates an array that flags potential key locations.
            \item \textbf{Training:} A Random Forest algorithm is trained on 128-byte slices of the preprocessed heap. The dataset is imbalanced, with the slices that contain keys being rare. A stacked classifier approach is used, comprising a high precision classifier and a high recall classifier.
            \item \textbf{Key Identification:} The machine learning model is used to predict which 128-byte slices of the heap dump are likely to contain encryption keys. These slices are then subjected to a brute-force method to actually extract the keys.
        \end{enumerate}
        
    SmartKex is significantly faster than the brute-force method alone and has a high degree of accuracy in identifying encryption keys. It also allows for the heap size to be reduced to less than 2\% of its original size, further optimizing the extraction process.

    SmartKex has broad applications in the field of cybersecurity, particularly in memory forensics. Its machine learning-assisted methodology can be adapted for other types of sensitive data extraction, making it a versatile tool for researchers and practitioners alike. The project is open-source, with the code available on GitHub \footnote{\url{https://github.com/smartvmi/Smart-and-Naive-SSH-Key-Extraction}}.

    \subsection{OpenSSH memory dumps dataset}

    SmartKex has contributed to the research community by generating a comprehensive annotated dataset of OpenSSH heap memory dumps \cite{SmartKex22}. The dataset is publicly available on Zenodo \footnote{\url{https://zenodo.org/record/6537904}}. 

    \begin{minipage}{\dimexpr\linewidth-20pt}
        The dataset is organized into two top-level directories: $Training$ and $Validation$ with an additional $Performance\_Test$. The first two main directories are further divided based on the SSH scenario, such as immediate exit, port-forward, secure copy, and shared connection. Each of these subdirectories is then categorized by the software version that generated the memory dump. Within these, the heaps are organized based on their key lengths, providing a multi-layered structure that aids in specific research queries.

        \begin{figure}[H]
            \centering
            \caption{Illustration of the Dataset Directory Structure}
            \label{fig:dataset_structure}
            \begin{minipage}{0.6\textwidth}
            \dirtree{%
            .1 /.
            .2 Performance\_Test.
            .3 V\_7\_1\_P1.
            .4 16.
            .4 24.
            .4 32.
            .3 ....
            .2 Training.
            .3 basic.
            .4 V\_6\_0\_P1.
            .5 16.
            .5 24.
            .5 32.
            .4 ....
            .3 ....
            .2 Validation.
            .3 basic.
            .4 V\_6\_0\_P1.
            .5 16.
            .5 24.
            .5 32.
            .4 ....
            .3 ....
            }
        \end{minipage}
        \end{figure}
    \end{minipage}

    Two primary file formats are used to store the data: JSON and RAW. The JSON files contain meta-information like the encryption method, virtual memory address of the key, and the key's value in hexadecimal representation. The RAW files, on the other hand, contain the actual heap dump of the OpenSSH process.

    \begin{minipage}{\dimexpr\linewidth-20pt}
        Here is an example of content of a RAW memory dump file, displayed using \textit{vim} and \textit{xdd} commands:

        \begin{lstlisting}[style=hexdump, caption={Hex Dump from \textit{Training/basic/V\_7\_8\_P1/16/5070-1643978841-heap.raw}}]
            00000000: 0000 0000 0000 0000 5102 0000 0000 0000  ........Q.......
            00000010: 0607 0707 0707 0303 0200 0006 0401 0206  ................
            00000020: 0200 0001 0100 0107 0604 0100 0000 0203  ................
            00000030: 0103 0101 0000 0000 0000 0000 0000 0002  ................
            00000040: 0001 0000 0000 0000 0000 0100 0000 0001  ................
            00000050: 8022 1a3a 3456 0000 007f 1a3a 3456 0000  .".:4V.....:4V..
            00000060: f040 1a3a 3456 0000 9032 1a3a 3456 0000  .@.:4V...2.:4V..
            00000070: 608b 1a3a 3456 0000 9047 1a3a 3456 0000  `..:4V...G.:4V..
        \end{lstlisting}
    \end{minipage}

    The original file contains the raw byte content of the heap dump of a specific version of OpenSSH. It is a binary file, which means that it is not human-readable. However, it can be converted to a human-readable format using the \textit{xxd} command. The first column to the left represents the offset in hexadecimal. The last column represents the actual content of the bytes, in ASCII format. The columns in between represent the content of the bytes in hexadecimal format.

    Since hexadecimal is a base-16 number system, each byte is represented by two hexadecimal digits. The ASCII representation of the bytes is displayed on the right, and is only used for reference, as it is not always possible to convert the bytes to ASCII. For instance, the bytes at offset 0x10 are not printable characters, and thus cannot be converted to ASCII. Each line represents 16 bytes, and the offset is incremented by 16 for each line.

    For the purpose of this thesis, it will be more interesting to visualize the content of the heap dump as 8 bytes lines. This can be achieved by using the \textit{xxd} command with the \textit{-c} option, as shown in the following example:

    \begin{minipage}{\dimexpr\linewidth-20pt}
        The same example as before, a memory dump file, displayed using \textit{vim} and \textit{xdd -c 8} commands:

        \begin{lstlisting}[style=hexdump, caption={Hex Dump}]
            00000000: 0000 0000 0000 0000  ........
            00000008: 5102 0000 0000 0000  Q.......
            00000010: 0607 0707 0707 0303  ........
            00000018: 0200 0006 0401 0206  ........
            00000020: 0200 0001 0100 0107  ........
            00000028: 0604 0100 0000 0203  ........
            00000030: 0103 0101 0000 0000  ........
            00000038: 0000 0000 0000 0002  ........
            00000040: 0001 0000 0000 0000  ........
            00000048: 0000 0100 0000 0001  ........
            00000050: 8022 1a3a 3456 0000  .".:4V..
            00000058: 007f 1a3a 3456 0000  ...:4V..
            00000060: f040 1a3a 3456 0000  .@.:4V..
            00000068: 9032 1a3a 3456 0000  .2.:4V..
            00000070: 608b 1a3a 3456 0000  `..:4V..
            00000078: 9047 1a3a 3456 0000  .G.:4V..
        \end{lstlisting}
    \end{minipage}

    This example shows the exact content of the preceding one. 

    To this RAW file is associated a JSON file, which contains its annotations.  

    \begin{minipage}{\dimexpr\linewidth-20pt}
         Here is a example of content of a JSON annotation file that comes with the previous RAW file:

        %\begin{lstlisting}[language=json, , ]
        \begin{lstlisting}[style=json, caption={Complete JSON example, from \textit{Training/basic/V\_7\_8\_P1/16/5070-1643978841.json}}, label={lst:json-annotation-ex-1}]
            {
                "SSH_PID": "5070",
                "SSH_STRUCT_ADDR": "56343a1a4800",
                "session_state_OFFSET": "0",
                "SESSION_STATE_ADDR": "56343a1a8d30",
                "newkeys_OFFSET": "344",
                "NEWKEYS_1_ADDR": "56343a1aaa40",
                "NEWKEYS_2_ADDR": "56343a1aab40",
                "enc_KEY_OFFSET": "0",
                "mac_KEY_OFFSET": "48",
                "name_ENCRYPTION_KEY_OFFSET": "0",
                "ENCRYPTION_KEY_1_NAME_ADDR": "56343a1a9db0",
                "ENCRYPTION_KEY_1_NAME": "aes128-gcm@openssh.com",
                "ENCRYPTION_KEY_2_NAME_ADDR": "56343a1a3fb0",
                "ENCRYPTION_KEY_2_NAME": "aes128-gcm@openssh.com",
                "key_ENCRYPTION_KEY_OFFSET": "32",
                "key_len_ENCRYPTION_KEY_OFFSET": "20",
                "iv_ENCRYPTION_KEY_OFFSET": "40",
                "iv_len_ENCRYPTION_KEY_OFFSET": "24",
                "KEY_A_ADDR": "56343a1a3170",
                "KEY_A_LEN": "12",
                "KEY_A_REAL_LEN": "12",
                "KEY_A": "feb5fd4ef0759b034d69b858",
                "KEY_B_ADDR": "56343a1a33e0",
                "KEY_B_LEN": "12",
                "KEY_B_REAL_LEN": "12",
                "KEY_B": "f50b988297fa19709445c4ee",
                "KEY_C_ADDR": "56343a1aa1b0",
                "KEY_C_LEN": "16",
                "KEY_C_REAL_LEN": "16",
                "KEY_C": "f5b53280e944db0fe196668d877cd4c0",
                "KEY_D_ADDR": "56343a1a4010",
                "KEY_D_LEN": "16",
                "KEY_D_REAL_LEN": "16",
                "KEY_D": "ac4f18a963d9e72c857497b7dc9d088d",
                "KEY_E_ADDR": "56343a1a7d90",
                "KEY_E_LEN": "0",
                "KEY_E_REAL_LEN": "0",
                "KEY_E": "",
                "KEY_F_ADDR": "56343a1a2f60",
                "KEY_F_LEN": "0",
                "KEY_F_REAL_LEN": "0",
                "KEY_F": "",
                "HEAP_START": "56343a198000"
            }
        \end{lstlisting}
    \end{minipage}

    Those annotation files contain the meta-information about the heap dump, such as the encryption method, virtual memory address of the key, and the key's value in hexadecimal representation. Those annotations are invaluable for the development of machine learning models for key extraction. 
    
    However, it is worth noting that the dataset is not perfect, as some of the annotations are sometimes missing. This is probably due to the fact that the annotations were generated automatically, and some of the keys were not correctly identified. For instance, in \textit{Training/basic/V\_7\_8\_P1/16/}, literally the first file of the dataset contains an incorrect annotation file, as some of the keys are missing. This is a limitation of the dataset that should be kept in mind when using it for research purposes.

    \begin{minipage}{\dimexpr\linewidth-20pt}
        Here is an example of content of a JSON annotation file with missing keys:

        \begin{lstlisting}[style=json, caption={Missing keys in JSON annotation file \textit{Training/basic/V\_6\_0\_P1/16/24375-1644243522.json}}]
            {
                "ENCRYPTION_KEY_NAME": "aes128-ctr",
                "ENCRYPTION_KEY_LENGTH": "16",
                "KEY_C": "689e549a80ce4be95d8b742e36a229bf",
                "KEY_D": "76788e66a56d2b61eec294df37422fcb",
                "HEAP_START": "5589d41e0000"
            }
        \end{lstlisting}
    \end{minipage}   

    The dataset is not just limited to SSH key extraction; it also serves as a resource for identifying essential data structures that hold sensitive information. This makes it a versatile tool for various research applications, including but not limited to machine learning models for key extraction and malware detection.

\section{Graph-based memory modelization}\label{sec:background:graph}

    In the following section, we present important concepts that will be used for the memory modelization of the heap dump.

    Because the dataset used is composed of RAW heap dump files from OpenSSH, it is a critical aspect to understand how memory works at a low level point of view. This section aims to provide an in-depth understanding of how memory is managed in C, the language used in the OpenSSH implementation of SSH, for a \textit{linux-x86\_64} architecture. We will explore the fundamental concepts of memory management, including the heap and the stack, memory allocation, and the role of pointers. These concepts will serve later as the foundation for our graph-based approach to memory modelization.

    This section will also introduce many graph theory and \acrfull{kg} concepts. We will explore the fundamentals of graph theory, including the definition of a graph, its components, and its properties. We will also discuss the concept of a \acrfull{kg}, which is a type of graph that stores information in the form of nodes and edges, and its applications in the field of machine learning.

    \subsection{Defining memory concepts and modelization}
    Memory management in C is a complex task that requires a deep understanding of the language's features, the operating system's capabilities and the compiler used. In C, memory is primarily managed through two built-in functions: \texttt{malloc} (memory allocation) and \texttt{free} (memory deallocation). These functions operate on two primary types of memory: the heap and the stack.

    \begin{itemize}
        \item \textbf{Heap vs Stack:} The heap is used for dynamic memory allocation, where variables are allocated and freed at runtime. In contrast, the stack is used for static memory allocation, where variables are allocated and deallocated automatically. The stack is faster but has a limited size, while the heap is more flexible but requires manual management to prevent memory leaks.
        
        \item \textbf{Heap Dump:} A heap dump is a snapshot of the heap's state at a given time. It provides valuable information about the memory layout, active pointers, and data stored in the heap. Analyzing heap dumps can help in debugging memory-related issues and understanding the program's behavior.
        
        \item \textbf{Memory Addresses:} Each location in memory is identified by a unique memory address. These addresses are usually represented in hexadecimal notation. Note that the address $0x0$ is reserved for the \textit{NULL} pointer, which is used to indicate that a pointer does not point to any memory location.

        \item \textbf{Pointer:} A pointer is a variable that stores the memory address of another variable. It is used to indirectly access the value of the variable it points to. Pointers are used extensively in C, particularly for dynamic memory allocation.
        
        \item \textbf{Data Structure:} A data structure is a collection of data values, the relationships among them, and the functions or operations that can be applied to the data. In the context of C programming, data structures are declared using the keyword \texttt{struct} and are byte aligned. This means that the size of the data structure is always a multiple of the size of the largest member of the structure. Structures can be nested within each other, and pointers can be used to indirectly access the members of a structure. Data structures are often stored in the heap using \texttt{malloc}.
        
        \item \textbf{Malloc headers:} When \texttt{malloc} is called, it allocates a block of memory in the heap and returns a pointer to the first byte of the block. The heap manager keeps track of these allocations through metadata, often stored in headers preceding the allocated blocks. These headers contain information such as the size of the allocated block and whether it is free or occupied. Note that the pointer returned by malloc in C points to the first byte of the block of memory that has been allocated for your use, not to the malloc header. The malloc header, is managed internally by the memory allocator and is not exposed to the programmer, but is visible in the heap dump.
        
    \end{itemize}

    \subsubsection{Endianness}
    
    Endianness refers to the byte order used to represent multi-byte data types. In a \textit{little-endian} system, the least significant byte is stored first, while in a \textit{big-endian} system, the most significant byte is stored first. Knowing the endianness of the system is crucial for interpreting the content of memory \cite{InferenceEndianness17}.
    
    For instance, the hexadecimal value \textit{0x56343a198000} (taken from \textit{"HEAP\_START"} of \ref{lst:json-annotation-ex-1}) is represented as $550179058774$ ($\simeq 5.50e+11$) in decimal basis in a little-endian system, while it is represented as $94782313037824$ ($\simeq 9.48e+13$) in a big-endian system.

    \begin{minipage}{\dimexpr\linewidth-20pt}
        \paragraph{Little-Endian Conversion}
        The conversion of a hexadecimal number in \textit{little-endian} format to a decimal number is given by the following formula:
        \par % solve the underfull hbox warning
        
        \vspace{2em}  % Adds vertical space
        \begin{center}
            $
            \text{Decimal} = \sum_{i=0}^{N-1} \left( \text{HexDigit}_{i} \times 16^{i} \right)
            $
        \end{center}
        \vspace{1em}
    \end{minipage}
    
    \begin{minipage}{\dimexpr\linewidth-20pt}
        \paragraph{Big-Endian convertion}
        And the convertion of a hexadecimal number in \textit{big-endian} format to a decimal number is given by following formula:
        \par % solve the underfull hbox warning
        
        \vspace{2em}  % Adds vertical space
        \begin{center}
            $
            \text{Decimal} = \sum_{i=0}^{N-1} \left( \text{HexDigit}_{N-1-i} \times 16^{i} \right)
            $
        \end{center}
        \vspace{1em}
    \end{minipage}

    Here, $ \text{HexDigit}_{i} $ is the value of the \(i\)-th digit in the little-endian hexadecimal number, and \( N \) is the number of digits in the hexadecimal number. Note that $ \text{HexDigit}_{i} $ should be converted to its decimal equivalent ('A' becomes 10, 'B' becomes 11, etc.) before performing the calculation.

    These formulas will be used later to convert pointer addresses from hexadecimal to decimal format in \textit{mem2graph}.

    \subsubsection{The role of entropy in forensic analysis}

    Entropy plays a pivotal role in forensic analysis, particularly in the context of memory dumps analysis. It serves as a measure of uncertainty and randomness, which can be crucial for tasks such as endianness detection and identifying encrypted keys in memory.

    As defined by Shannon in the realm of information theory, it is a measure of the uncertainty or randomness associated with a set of possible outcomes \cite{InferenceEndianness17}. In digital applications, when calculated using the logarithm to base 2, entropy represents the amount of bits of information in a message.

    \begin{minipage}{\dimexpr\linewidth-20pt}
        \paragraph{Entropy Formula}
        The entropy \( H \) of a message is calculated using the formula:
        \par % solve the underfull hbox warning
        
        \vspace{2em}  % Adds vertical space
        \begin{center}
            $
            H = - \sum_{i=1}^{n} p_i \log_2 p_i
            $
        \end{center}
        \vspace{1em}

        Where $ p_1, p_2, \ldots, p_n $ are the probabilities of the set of all possible messages \cite{InferenceEndianness17}.
    \end{minipage}

    Endianness, as presented before, refers to the byte order used to represent multi-byte data types in computer memory \cite{InferenceEndianness17}. It is necessary to know the endianness of the heap dump to correctly interpret the content of memory, and especially the addresses of potential pointers. In this context, entropy can be used to infer the endianness of a system by analyzing the distribution of byte values in a memory dump, as presented in \citetitle{InferenceEndianness17} \cite{InferenceEndianness17}.

    Entropy is also a key element for encryption key detection, since those should be random byte sequences with high entropy \cite{SmartKex22}. By examining 8-byte aligned data for high entropy, it is possible to detect potential keys in a heap dump. Techniques such as the calculation of discrete differences and logical operations can further refine this detection as described in \citetitle{SmartKex22} \cite{SmartKex22}.

    \subsection{Graphs and Knowledge Graphs}
    In this project, we (i.e. Clément Lahoche and the author) have built a program called \textit{mem2graph}, a generic tool developed in Rust for efficiency. It is used to convert a RAW memory dump into a graph representation. This graph is technically a directed Edge-labeled heterogeneous property graph, whose concepts are described later. As such, and while it is debatable whether or not \textit{mem2graph} can be seen as building Knowledge Graphs (KGs), it relies on many concepts from the domain, necessitating a proper introduction.

    \subsubsection{Defining Graph Theory concepts}
    Graph theory is a mathematical field concerned with the study of graphs. It has applications in various fields, including computer science, social sciences, or linguistics. Graphs are used to model pairwise relations between objects, and the study of graphs involves analyzing the properties of these relations. In a few words, a graph is just a collection of nodes and edges.
    
    A graph can be formally defined  as: \say{a pair \( G = (S, A) \) where:
    \begin{itemize}
        \item \( S \) is a finite set of vertices.
        \item \( A \) is a set of pairs of vertices \((s_i, s_j) \in S^2\).
    \end{itemize}
    Graphs can be either directed or undirected. In a directed graph, the pairs \((s_i, s_j)\) are ordered, representing arcs from \(s_i\) to \(s_j\). In an undirected graph, the pairs are unordered, representing edges between \(s_i\) and \(s_j\) } \cite{GraphTheorySolnon}. Let's introduce some vocabulary to describe graphs:

    \begin{itemize}
        \item \textbf{Node (or Vertex)}: A single entity in the graph, often represented as a circle.
        \item \textbf{Edge (or Arc)}: A connection between two nodes, often represented as a line or arrow.
        \item \textbf{Degree}: The number of edges connected to a node.
        \item \textbf{Path}: A sequence of edges that connect two nodes.
        \item \textbf{Cycle}: A path that starts and ends at the same node.
        every pair of nodes.
        \item \textbf{Order}: The number of vertices in the graph.
        \item \textbf{Adjascency}: Two nodes are adjacent if there is an edge between them.
        \item \textbf{Loop}: An edge that connects a node to itself.
        \item \textbf{Weight}: A value assigned to an edge.
        \item \textbf{Ancestors (parents) and Descendants (children)}: A node \(s_i\) is an ancestor of \(s_j\) if there is a path from \(s_i\) to \(s_j\). A node \(s_j\) is a descendant of \(s_i\) if there is a path from \(s_i\) to \(s_j\).
    \end{itemize}
    
    Various other terminologies and concepts exist in graph theory, but these are the most important ones for our purposes. For a more in-depth understanding of graph theory, the reader is encouraged to consult the work by \citeauthor{GraphTheorySolnon} as a quick introduction, \cite{GraphTheorySolnon} or a more in-depth one by \citeauthor{GraphTheoryIntro01} \cite{GraphTheoryIntro01}.

    \subsubsection{Graphs types}
    Graphs offer a flexible way to conceptualize, represent, and integrate diverse and incomplete data. Many graph models exist, each with its own advantages and disadvantages, as well as graph properties \footnote{Some parts of the following section are directly from a prior work for Seminar 5369S: Knowledge Graphs, written during summer 2023. As of the date of writting and to the best of my knowledge, this work has not been published, and as such, cannot be properly referenced.}. Those different types of graphs include:

    \begin{itemize}
        \item \textbf{Directed Edge-labelled Graphs (DEL):} The classic graph, set of nodes and edges that connect the nodes with in certain way. \acrshort{rdf} is a popular \acrshort{del} data model.
        \item \textbf{Heterogeneous Graphs:} Each node and edge is assigned one type, allowing for partitioning nodes according to their type, which is useful for machine learning.
        \item \textbf{Property Graphs:} Allows a set of property-value pairs and a label to be associated with nodes and edges. This model is used in Neo4j and offers great flexibility but is harder to handle and query.
        \item \textbf{Graph Dataset:} A set of named graphs, with a default graph with no ID. Useful when working with different sources.
        \item \textbf{Hypergraphs:} Allow edges that connect sets rather than pairs of nodes.
    \end{itemize}

    A given graph can have a range of properties that give some insights into its structure. Here are some important properties of graphs:

    \begin{itemize}
        \item \textbf{Connected Graph}: A graph is connected if there is a path between every pair of nodes.
        \item \textbf{Disconnected Graph}: A graph is disconnected if there is at least one pair of nodes that are not connected by a path.
        \item \textbf{Cyclic Graph}: A graph is cyclic if it contains at least one cycle.
        \item \textbf{Acyclic Graph}: A graph is acyclic if it does not contain any cycles.
        \item \textbf{Complete Graph}: A graph is complete if there is an edge between every pair of nodes.
    \end{itemize}

    \subsubsection{Graph vs Knowledge Graph}
    A \acrfull{kg} is a specialized form of graph intended to accumulate and convey real-world knowledge. As said before, we are not technically building KGs, but we rely on many concepts from the domain. Research on \acrshort{kg} has further accelerated in recent years, and introduced significant improvement to Graph Theory, especially in the practical applications of graph construction and use for Machine Learning or Deep Learning \cite{KG21}. They have a number of benefits when compared with a relational model or NoSQL alternatives, such as the ability for data to evolve in a more flexible manner, and the capacity to organize data in a way that is not hierarchical. They can represent incomplete information, and does not require a precise schema \cite[p.2]{KG21}, which is invaluable in the context of memory analysis, where the structure of the heap is not known in advance.

    While all KGs are graphs, not all graphs are KGs. However the line between the two is often blurry, and the distinction is not always clear. The term "knowledge graph" first appeared in 1973, but really gained popularity through a 2012 blog post about Google's Knowledge Graph \cite*{googleblog2023knowledgegraph}. Several definition attempts have been made, but none of them are universally accepted. Below are listed some of the most common definitions of Knowledge Graphs:
    \begin{itemize}
        \item \say{A graph of data intended to accumulate and convey knowledge of the real world, whose nodes represent entities of interest and whose edges represent potentially different relations between these entities.} \cite{KG21}
        \item \say{A graph of data consisting of semantically described entities and relations of different types that are integrated from different sources. Entities have a unique identifier. KG entities and relations are semantically described using an ontology or, more clearly, an ontological representation.} \cite{CKG23}
    \end{itemize}
    
    In KGs, edges are often labeled and may represent complex relationships like "is a subclass of" or "is married to," allowing for more expressive power. The very nature of KG makes any definition attempt difficult. Indeed, KG is a broad concept that can be applied to many different domains, use cases and can have diverse implementations. The definition of KGs is thus very context-dependent, and it's debatable where the line between a graph and a KG is drawn. 
    
    For the purpose of this thesis, we won't focus on this distinction and just consider that we deal with complex \textit{graphs}, which are graphs that are not necessarily KGs, but that can be used to represent complex relationships between entities and be leveraged using \acrshort{kg}-related tecnhiques for advance tasks like feature engineering, automated embedding, inductive reasoning and learning.

    \subsubsection{Inductive Reasoning and Learning}
    Inductive reasoning in Knowledge Graphs (KGs) involves techniques like embedding and Graph Convolutional Networks (GCNs) to learn the potential underlying structure of the graph. This is particularly useful for tasks like link prediction, node classification, and clustering.

    For the sake of clarity and grouping, we will present the concepts of graph embedding and GCNs in the next sections, but it is important to note that they are not mutually exclusive. In fact, GCNs can be used to generate embeddings, and embeddings can be used as input for GCNs. As such, they are often used together in the context of KGs, or in our case, complex graphs.

\section{Data processing for Machine Learning}\label{sec:background:processing}

    \subsection{Feature engineering}
    * what is feature engineering
    * why is it useful

    \subsection{Embeddings vs Feature engineering}
    * what are embeddings
    * difference between the two
    * when to use one or the other

    \subsection{Graph-based embeddings}
    For the Graph Convolution Network part, we want to prepare the memory graph into a meaningful embedding.
    * what techniques can be used for automatic embedding of the heap memory graph
    * present some python libraries that can be used to embed the graph
    * present the mathematical concepts involved, and their associated formulas

    Graph embedding techniques aim to map nodes and edges in a graph to vectors in a low-dimensional space. The primary goal is to preserve the graph's structural properties, such as node connectivity and community structure, in the embedded space. These vectors can then be used for various machine learning tasks like clustering, classification, and link prediction. Here are some common and advanced techniques:
    
    \begin{itemize}
        \item \textbf{Node2Vec}: This algorithm learns continuous feature representations for nodes by optimizing a neighborhood-preserving objective. It employs biased random walks and uses the Skip-gram model to generate embeddings.
        
        \item \textbf{DeepWalk}: Similar to Node2Vec, DeepWalk uses random walks to generate node sequences. It employs the Skip-gram model from natural language processing to learn embeddings. Unlike Node2Vec, it does not use biased walks.
        
        \item \textbf{Spectral Clustering}: This technique is based on the spectral theory of graphs. It utilizes the eigenvalues and eigenvectors of the Laplacian matrix of the graph to find an optimal embedding. Spectral Clustering is particularly useful for community detection.
        
        \item \textbf{LINE}: Large-scale Information Network Embedding (LINE) aims to preserve both local and global network structures. It optimizes two objectives: first-order and second-order proximities between nodes.
        
        \item \textbf{GloVe on Graphs}: An adaptation of the Global Vectors for Word Representation (GloVe), this method constructs a co-occurrence matrix based on node neighborhoods and then factorizes it to obtain embeddings.
        
        \item \textbf{Graph Factorization}: This method directly factorizes the adjacency matrix of the graph to learn node embeddings, making it computationally efficient but less capable of capturing complex structures.
        
        \item \textbf{SDNE (Structural Deep Network Embedding)}: SDNE employs a deep autoencoder to learn complex and non-linear node embeddings while preserving first-order and second-order proximities.
    \end{itemize}

    \subsection{Dataset splitting and sampling}
    * Explain why it is necessary
    * present the different techniques, like folds and the associated python libraries (like scikit-learn) that can be used to do that.

\section{Machine Learning and Deep Learning}\label{sec:background:ml}
    \subsection{Machine Learning}
    Introduce the importance of ML in today's research and breakthroughts
    * what is ML
    * how does it works
    * model evaluation, concepts like precision, recall, F-1 score and so forth

    \subsection{Machine Learning models}
    binary classification problem: either a block is the first block of a key, or it is not. Different ML models are suitable for the task (we are using Python Sciki-learn):

    * present the different models, pros and cons, that we can use, the maths and formulas behind

    \subsection{Deep Learning and GCN}
    We can also build our own Deep Learning neural network using Pytorch, specifically when dealing with automatically generated embeddings from the graph representation of the heap dump

    * Present what is deep learning and neural Networks
    * what are GCN
    * present different architectures (python libraries that can be used to embed the grap and generate GCN for binary predicting key nodes)

    \subsubsection{Graph Convolutional Networks (GCNs)}
    Graph Convolutional Networks extend the concept of convolution from images to graphs. They are designed to work with non-Euclidean data and are particularly useful for semi-supervised learning tasks on graphs. GCNs aim to learn a function that maps nodes to a low-dimensional space while considering their local neighborhood and features. Below are some common and advanced techniques:

    \begin{itemize}
        \item \textbf{Vanilla GCN}: The basic GCN model consists of an input layer, one or more hidden layers, and an output layer. Each layer is associated with a graph convolution operation that updates the node features based on their neighbors.
        
        \item \textbf{GraphSAGE (Graph Sample and Aggregation)}: This model generalizes the GCN framework by allowing various aggregation functions like mean, LSTM, and pooling to combine information from a node's neighbors.
        
        \item \textbf{ChebNet}: ChebNet uses Chebyshev polynomials to generalize the convolution operation in the spectral domain. This allows the model to capture a broader range of graph structures.
        
        \item \textbf{GAT (Graph Attention Networks)}: GAT introduces attention mechanisms into GCNs, enabling the model to weigh neighbors differently when aggregating information.
        
        \item \textbf{MoNet}: This model employs a mixture model to generalize the convolution operation, making it capable of handling graphs with diverse structures.
        
        \item \textbf{Graph Isomorphism Network (GIN)}: GIN is designed to capture the isomorphism between different graphs, making it powerful for tasks like graph classification.
        
        \item \textbf{Cluster-GCN}: A scalable method that divides the graph into clusters and performs mini-batch training, making it suitable for large graphs.
        
        \item \textbf{ARMA (AutoRegressive Moving Average)}: This model combines autoregressive models and moving averages to capture both spatial and temporal dependencies in graph data.
    \end{itemize}



