\section{Introduction}\label{chap:introduction}

% Motivate your research and outline the research gap in this chapter. Why is your thesis relevant and what do you address, what has not been addressed before. 

% ## General Requirements to the thesis:
% * 60 pages of content in this format. Content does not include table of content, lists, appendices etc.
% * Proper scientific referencing
% * Introduction and Background should be less than 50\% of the thesis
% * Images should be readable and in the proper size. 

The digital age has brought with it an unprecedented increase in the volume and complexity of data, making cybersecurity a critical focus area. This evolving landscape is fraught with challenges that continue to amplify the importance of digital forensics in IT systems. One area that stands out for its complexity and importance is the Secure Shell protocol (SSH), particularly its popular implementation through OpenSSH. While SSH plays an indispensable role in secure communications, it also presents a unique set of challenges, most notably the concealment of malicious activities.

A common case is when a unauthorized actor gains access to SSH keys so as to get access to a system. This can happen through a malicious human actor, but more commonly through automated processes such as malwares and botnets. This situations present a formidable and growing threat to cybersecurity, affecting a broad range of stakeholders from governments and financial institutions to individual users. For botnets alone, and only considering a single year like 2019, the number of Command and Control (C\&C) servers surged by 71.5\%, and the 3ve botnet caused an estimated \$19 billion in advertising theft \cite{SSHBotnetInfect21}. Many malwares and botnets \say{have in common that they have used as attack vector the Secure Shell (SSH) remote access service} \cite{SSHBotnetInfect21}. 

At the heart of the issue lies the fact that SSH veils its communications through encryption, making it difficult to detect malicious activities. To be able to detect those potential malicious actors, it is possible to replace SSH by a honeypot that enable to monitor pseudo-SSH activities. There is a range of readily available honeypots, such as Kippo or Cowrie, which are designed to emulate a vulnerable SSH system and attract attackers \cite{ClassificationMalware21}. The problem lies that thoses honeypots are not able to mimic perfectly a real system, which makes them easy to detect by experienced attackers. As stated by \citetitle{SSHHoneypotEffectiveness23}: \say{The ability to collect meaningful malware from attackers depends on how the attackers receive the honeypot. Most attackers fingerprint targets before they launch their attack, so it would be very beneficial for security researchers to understand how to hide honeypots from fingerprinting and trick the attackers into depositing malware. [...] What is certain is that if a cautious attacker believes they are in a honeypot, they will leave without depositing malware onto the system, which reduces the effectiveness of the honeypot} \cite{SSHHoneypotEffectiveness23}. 

Instead of relying on softwares that mimics a real system, it is possible to use a real system directly. Since SSH encryption keys are typically stored in the main memory of a system, it is possible for the administrators to extract them through the exploitation of memory dumps of a targeted system. In this context, the ability to detect SSH keys in memory dumps, and specifically OpenSSH keys, is critical to the development of effective SSH honeypot-like systems. The research introduced by the SmartVMI project with SSHKex, SmartKex, the present thesis and the future related work could be used to develop a such a new type of system-monitoring. Such a honeypot would be very difficult to detect by the attacker, which would increase its effectiveness. The present report is focused on the SSH key detection in memory dumps, which is a key component allowing to decode SSH communications such that it become possible to intercept malicious communications and to detect malicious activities.

\section{Research Questions}

% Write down and explain your research questions (2-5)

The initial objective of this thesis is to answer the following research questions:
\begin{itemize}
	\item RQ1: What is the state of the art in the field of security key detection in heap dump memory?
	\item RQ2: What are the challenges in the field of security key detection in heap dump memory?
\end{itemize}

\section{Structure of the Thesis}

% Explain the structure of the thesis. 

\begin{comment}
\section{Example citation \& symbol reference}\label{sec:citation}
For symbols look at \cite{latex_symbols_2017}.


\section{Example reference}
Example reference: Look at chapter~\ref{chap:introduction}, for sections, look at section~\ref{sec:citation}.

\section{Example image}

\begin{figure}
	\centering
	\includegraphics[width=0.5\linewidth]{uni-logo}
	\caption{Meaningful caption for this image}
	\label{fig:uniLogo}
\end{figure}

Example figure reference: Look at Figure~\ref{fig:uniLogo} to see an image. It can be \texttt{jpg}, \texttt{png}, or best: \texttt{pdf} (if vector graphic).

\section{Example table}

\begin{table}
	\centering
	\begin{tabular}{lr}
		First column & Number column \\
		\hline
		Accuracy & 0.532 \\
		F1 score & 0.87
	\end{tabular}
	\caption{Meaningful caption for this table}
	\label{tab:result}
\end{table}

Table~\ref{tab:result} shows a simple table\footnote{Check \url{https://en.wikibooks.org/wiki/LaTeX/Tables} on syntax}
\end{comment}

