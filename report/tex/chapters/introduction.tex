\chapter{Introduction}\label{chap:introduction}

% Motivate your research and outline the research gap in this chapter. Why is your thesis relevant and what do you address, what has not been addressed before. 

% ## General Requirements to the thesis:
% * 60 pages of content in this format. Content does not include table of content, lists, appendices etc.
% * Proper scientific referencing
% * Introduction and Background should be less than 50\% of the thesis
% * Images should be readable and in the proper size. 

The digital age has brought with it an unprecedented increase in the volume and complexity of data that is being generated, stored, and processed. This data is often sensitive in nature, and its security is of paramount importance, making cybersecurity a critical focus area. This evolving landscape is fraught with challenges that continue to amplify the importance of digital forensics in IT systems. One area that stands out for its widespread use and importance is the Secure Shell protocol (SSH) and its most popular implementation, OpenSSH. SSH is a cryptographic network protocol widely used for secure remote access to systems. It is also used for secure file transfer, and as a secure tunnel for other applications. SSH is a key component of IT systems whose encryption capabilities are critical to the security of IT systems. However, it also presents a unique set of challenges, most notably the concealment of malicious activities.

A common case is when an unauthorized actor gains access to SSH keys so as to get access to a system. This can happen through a malicious human actor, but more commonly through automated processes such as malwares and botnets. This situation presents a formidable and growing threat to cybersecurity, affecting a broad range of stakeholders from governments and financial institutions to individual users. In just 2019, the number of Command and Control (C\&C) servers for botnets increased by 71.5\%, leading to an estimated \$19 billion in advertising theft \cite{SSHBotnetInfect21}. Many malwares and botnets \say{have in common that they have used as attack vector the Secure Shell (SSH) remote access service} \cite{SSHBotnetInfect21}. 

At the heart of the issue lies the fact that SSH veils its communications through encryption, making it difficult to detect malicious activities. To be able to detect those potential malicious actors, it is possible to replace SSH by a honeypot that enables to monitor pseudo-SSH activities. There is a range of readily available honeypots, such as Kippo or Cowrie, which are designed to emulate a vulnerable SSH system and attract attackers \cite{ClassificationMalware21}. The problem lies that those honeypots are not able to mimic perfectly a real system, which makes them easy to detect by experienced attackers. As stated by \citetitle{SSHHoneypotEffectiveness23}: \say{The ability to collect meaningful malware from attackers depends on how the attackers receive the honeypot. Most attackers fingerprint targets before they launch their attack, so it would be very beneficial for security researchers to understand how to hide honeypots from fingerprinting and trick the attackers into depositing malware. [...] What is certain is that if a cautious attacker believes they are in a honeypot, they will leave without depositing malware onto the system, which reduces the effectiveness of the honeypot} \cite{SSHHoneypotEffectiveness23}. 

There are other approaches that allow to decrypt SSH connections without relying on a honeypot, like the \textit{man-in-the-middle} or \textit{binary manipulation} with their own set of challenges \cite{SSHkex22}. Instead of relying on softwares that mimics or modify a real system, it is possible to use a real unmodified system directly. The idea is to be able to decrypt SSH connection channels, which is possible if the SSH keys are known. Since SSH encryption keys are typically stored in the main memory of a system, it is possible for the administrators to extract them through the exploitation of memory dumps of a targeted system. In this context, the ability to detect SSH keys in memory dumps, and specifically OpenSSH keys, is critical to the development of effective SSH honeypot-like systems. The research introduced by the SmartVMI project with SSHKex, SmartKex, the present thesis and the future related work could be used to develop such a new type of system-monitoring tools. This new kind of tools would be very difficult to detect by attackers, increasing their effectiveness, and wouldn't require the alteration of the system. The present report is focused on the SSH key detection in memory dumps, which is a key component allowing to decode SSH communications such that it becomes possible to intercept malicious communications and to detect malicious activities.

\section{Research Questions}

% Write down and explain your research questions (2-5)

At the very beginning of this thesis, first questions were:
\begin{itemize}
	\item What is the state of the art in the field of security key detection in heap dump memory?
	\item What are the challenges of security key detection in heap dump memory?
	\item How can the existing methods for detecting SSH keys in OpenSSH heap dumps be improved?
\end{itemize}

The SmartVMI project has already made significant progress in the detection of SSH keys in OpenSSH heap dumps. An open dataset of memory dumps has been created, and a simple yet effective method for detecting SSH keys has been developed. The dataset has been used to train and test simple machine learning algorithms, and the results have been promising. The research has been published in the form of two papers, SSHkex \cite{SSHkex22} and SmartKex \cite{SmartKex22}, which is the basis of this thesis. 

However, there is still room for improvement, particularly in the area of machine learning algorithms. This thesis seeks to build upon the existing research by refining feature extraction techniques and exploring innovative methods for effective key detection prediction. The objective is to accurately predict the presence and location of SSH keys within memory dumps. Rooted in this context, this Masterarbeit aims to address several key research questions:

\begin{itemize}
	\item \textbf{Memory graph:} How can we develop a memory graph representation to improve the prediction of SSH keys in memory dumps?
	\item \textbf{Memory graph embedding:} How can we develop a memory graph embedding representation to improve the prediction of SSH keys in memory dumps?
	\item \textbf{Feature importance:} What features are most indicative of SSH keys in memory dumps?
	\item \textbf{Feature extraction:} How can these features be extracted from memory dumps and used to train machine learning algorithms?
	\item \textbf{\acrshort{ml} for key prediction:} How can machine learning algorithms be optimized for the prediction of SSH keys in memory dumps? 
	\item \textbf{\acrlong{gcn} for key prediction:} How can \acrshort{gcn} be used to improve the prediction of SSH keys in memory dumps, and how does it compare to traditional machine learning algorithms?
\end{itemize}

By tackling these research questions, this thesis seeks not only to advance the academic understanding of SSH key prediction and digital forensics but also to provide practical insights that could lead to the development of more secure and effective systems.

\section{Commitment to Open Science and Reproducibility}

In alignment with the principles of Open Science, this thesis aims to be not just a scholarly report but also a comprehensive guide for anyone who wishes to understand, replicate, or extend the work presented. Open Science is a movement that advocates for the transparent and accessible sharing of scientific research, data, and dissemination processes \cite{WhyNotShareData22}. It is built on six fundamental principles \cite{WasIstOpenScience23}:

\begin{enumerate}
    \item \textbf{Open Methodology}: Detailed methodologies are provided to ensure that the experiments can be replicated.
    \item \textbf{Open Source}: All code used in this research is available for scrutiny and reuse. As such, all code including the \LaTeX{} code for the present report \footnote{The present report repository can be found here: \url{https://github.com/0nyr/masterarbeit\_report}} is properly documented and can be accessed on GitHub. 
    \item \textbf{Open Data}: Raw data and the data processing techniques are made publicly available.
    \item \textbf{Open Access}: The research is published in a manner that is free for all to read and download.
    \item \textbf{Open Peer Review}: The peer review process is transparent. In the case of this Masterarbeit, the research is reviewed by the supervisors of the project.
    \item \textbf{Open Educational Resources}: Any educational content produced is shared openly.
\end{enumerate}

To ensure the highest level of reproducibility and accessibility, this thesis includes what might seem like exhaustive details, such as hardware or software specifications, precise shell commands and some code implementations used during the research. These are included to provide a complete picture and to minimize the friction for those who wish to replicate the experiments, whatever their level of expertise may be. By adhering to the principles of Open Science, this thesis aims to contribute to a more transparent, collaborative, and efficient scientific community.
	
	\subsection{GitHub Repositories}

	In the context of the present Masterarbeit, a number of GitHub repositories have been created to facilitate the sharing of code and data. These repositories are listed below:

	\begin{itemize}

		\item \textbf{masterarbeit\_report\_onyr}: Repository containing the LaTeX code for the report as well as several scripts related to dataset exploration: \url{https://github.com/passau-masterarbeit-2023/masterarbeit_report_onyr}
		
		\item \textbf{mem2graph}: Memory graph creation utility built in Rust, featuring different graph creation and embedding strategies. Collaboration with Clément Lahoche: \url{https://github.com/passau-masterarbeit-2023/mem2graph}

		\item \textbf{research-base}: Custom Python framework for developing programs that include all the basics of a research project, such as logging, environment and argument loading, result keeping, and more. Collaboration with Clément Lahoche: \url{https://github.com/0nyr/research-base}

		\item \textbf{data\_processing}: Python program for data processing and machine learning for SSH key prediction. This repository contains tests on machine learning model training and evaluation for classical .csv based embedding files from \textit{mem2graph}: \url{https://github.com/passau-masterarbeit-2023/data_processing}
		
		\item \textbf{phdtrack\_project\_3}: Legacy repository containing the first version of the memory graph creation utility and the first version of the dataset creation script. Collaboration with Clément Lahoche. \url{https://github.com/0nyr/phdtrack_project_3}
		
		\item \textbf{memory\_graph\_gcn}: Main Python program and scripts around GCN for SSH key prediction. This program leverages the modified DOT file with embedding generated by \textit{mem2graph}: \textit{mem2graph}:
		\url{https://github.com/passau-masterarbeit-2023/memory_graph_gcn}

		\item \textbf{phdtrack\_server\_scripts}: Scripts for the servers used for computing experiments. This repository contains the scripts used to install the necessary tooling and run the experiments on the different servers we used. Collaboration with Clément Lahoche:
		\url{https://github.com/passau-masterarbeit-2023/phdtrack_server_scripts}
	\end{itemize}

	As one can see, and considering the collaborative work effort that has been, it has been decided to regroup all repositories related to the OpenSSH heap dump exploration in a single GitHub organization, \textit{passau-masterarbeit-2023} \url{https://github.com/passau-masterarbeit-2023}.

	\section{Structure of the Thesis}

	% explain the structure of the thesis, what is in which chapter and why.

	The present thesis is organized in a manner that ensures a coherent and logical flow of information, following the standard structure of a Masterarbeit report. The structure is designed to gradually guide the reader from understanding the context and background of the research to the intricacies of the methods employed, and finally to the interpretation of the results. Below is a breakdown of each section:
	
	\begin{itemize}
		\item \textbf{Background Section:} This section serves as an introduction to the research context and establishes the foundation for the thesis. It outlines the previous work and state of the art, providing the reader with an understanding of existing knowledge and identifying gaps that this research aims to address. Key concepts, terminologies, and theories relevant to the study are introduced, setting the stage for the subsequent sections.
		
		% \item \textbf{Related Work Section:} Here, a detailed review of prior research pertinent to key extraction from memory dumps and the SmartVMI project is presented. This section delves into existing literature and studies, highlighting their methodologies, findings, and relevance to the current thesis.
		
		\item \textbf{Methods Section:} This section meticulously describes the methods and approaches employed during the research. From the creation of the dataset to the selection and implementation of machine learning algorithms, this section ensures that the research process is transparent and reproducible.
		
		\item \textbf{Results Section:} The results' section presents the data obtained from the experiments conducted, outlining both the layout of programs used and the raw results. It provides a factual account of the findings without delving into interpretation or discussion.
		
		\item \textbf{Discussion Section:} This section offers an analysis and interpretation of the results obtained. It explores the implications of the findings, discusses the limitations of the study, and contextualizes the results within the broader research landscape.
		
		\item \textbf{Conclusion:} The concluding section succinctly recall the salient points of the thesis. It underscores the contributions made to the field and suggests avenues for future research, providing a fitting closure to the report.
	\end{itemize}
	
	In structuring the thesis in this manner, the intention is to provide the reader with a comprehensive yet accessible insight into the research undertaken all along this year-long project. 
	

\begin{comment}
\section{Example citation \& symbol reference}\label{sec:citation}
For symbols look at \cite{latex_symbols_2017}.


\section{Example reference}
Example reference: Look at chapter~\ref{chap:introduction}, for sections, look at section~\ref{sec:citation}.

\section{Example image}

\begin{figure}
	\centering
	\includegraphics[width=0.5\linewidth]{uni-logo}
	\caption{Meaningful caption for this image}
	\label{fig:uniLogo}
\end{figure}

Example figure reference: Look at Figure~\ref{fig:uniLogo} to see an image. It can be \texttt{jpg}, \texttt{png}, or best: \texttt{pdf} (if vector graphic).

\section{Example table}

\begin{table}
	\centering
	\begin{tabular}{lr}
		First column & Number column \\
		\hline
		Accuracy & 0.532 \\
		F1 score & 0.87
	\end{tabular}
	\caption{Meaningful caption for this table}
	\label{tab:result}
\end{table}

Table~\ref{tab:result} shows a simple table\footnote{Check \url{https://en.wikibooks.org/wiki/LaTeX/Tables} on syntax}
\end{comment}

